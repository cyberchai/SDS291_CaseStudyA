% Options for packages loaded elsewhere
\PassOptionsToPackage{unicode}{hyperref}
\PassOptionsToPackage{hyphens}{url}
\PassOptionsToPackage{dvipsnames,svgnames,x11names}{xcolor}
%
\documentclass[
  letterpaper,
  DIV=11,
  numbers=noendperiod]{scrartcl}

\usepackage{amsmath,amssymb}
\usepackage{iftex}
\ifPDFTeX
  \usepackage[T1]{fontenc}
  \usepackage[utf8]{inputenc}
  \usepackage{textcomp} % provide euro and other symbols
\else % if luatex or xetex
  \usepackage{unicode-math}
  \defaultfontfeatures{Scale=MatchLowercase}
  \defaultfontfeatures[\rmfamily]{Ligatures=TeX,Scale=1}
\fi
\usepackage{lmodern}
\ifPDFTeX\else  
    % xetex/luatex font selection
\fi
% Use upquote if available, for straight quotes in verbatim environments
\IfFileExists{upquote.sty}{\usepackage{upquote}}{}
\IfFileExists{microtype.sty}{% use microtype if available
  \usepackage[]{microtype}
  \UseMicrotypeSet[protrusion]{basicmath} % disable protrusion for tt fonts
}{}
\makeatletter
\@ifundefined{KOMAClassName}{% if non-KOMA class
  \IfFileExists{parskip.sty}{%
    \usepackage{parskip}
  }{% else
    \setlength{\parindent}{0pt}
    \setlength{\parskip}{6pt plus 2pt minus 1pt}}
}{% if KOMA class
  \KOMAoptions{parskip=half}}
\makeatother
\usepackage{xcolor}
\usepackage[left=1in,right=1in,top=1in,bottom=1in]{geometry}
\setlength{\emergencystretch}{3em} % prevent overfull lines
\setcounter{secnumdepth}{-\maxdimen} % remove section numbering
% Make \paragraph and \subparagraph free-standing
\ifx\paragraph\undefined\else
  \let\oldparagraph\paragraph
  \renewcommand{\paragraph}[1]{\oldparagraph{#1}\mbox{}}
\fi
\ifx\subparagraph\undefined\else
  \let\oldsubparagraph\subparagraph
  \renewcommand{\subparagraph}[1]{\oldsubparagraph{#1}\mbox{}}
\fi

\usepackage{color}
\usepackage{fancyvrb}
\newcommand{\VerbBar}{|}
\newcommand{\VERB}{\Verb[commandchars=\\\{\}]}
\DefineVerbatimEnvironment{Highlighting}{Verbatim}{commandchars=\\\{\}}
% Add ',fontsize=\small' for more characters per line
\usepackage{framed}
\definecolor{shadecolor}{RGB}{241,243,245}
\newenvironment{Shaded}{\begin{snugshade}}{\end{snugshade}}
\newcommand{\AlertTok}[1]{\textcolor[rgb]{0.68,0.00,0.00}{#1}}
\newcommand{\AnnotationTok}[1]{\textcolor[rgb]{0.37,0.37,0.37}{#1}}
\newcommand{\AttributeTok}[1]{\textcolor[rgb]{0.40,0.45,0.13}{#1}}
\newcommand{\BaseNTok}[1]{\textcolor[rgb]{0.68,0.00,0.00}{#1}}
\newcommand{\BuiltInTok}[1]{\textcolor[rgb]{0.00,0.23,0.31}{#1}}
\newcommand{\CharTok}[1]{\textcolor[rgb]{0.13,0.47,0.30}{#1}}
\newcommand{\CommentTok}[1]{\textcolor[rgb]{0.37,0.37,0.37}{#1}}
\newcommand{\CommentVarTok}[1]{\textcolor[rgb]{0.37,0.37,0.37}{\textit{#1}}}
\newcommand{\ConstantTok}[1]{\textcolor[rgb]{0.56,0.35,0.01}{#1}}
\newcommand{\ControlFlowTok}[1]{\textcolor[rgb]{0.00,0.23,0.31}{#1}}
\newcommand{\DataTypeTok}[1]{\textcolor[rgb]{0.68,0.00,0.00}{#1}}
\newcommand{\DecValTok}[1]{\textcolor[rgb]{0.68,0.00,0.00}{#1}}
\newcommand{\DocumentationTok}[1]{\textcolor[rgb]{0.37,0.37,0.37}{\textit{#1}}}
\newcommand{\ErrorTok}[1]{\textcolor[rgb]{0.68,0.00,0.00}{#1}}
\newcommand{\ExtensionTok}[1]{\textcolor[rgb]{0.00,0.23,0.31}{#1}}
\newcommand{\FloatTok}[1]{\textcolor[rgb]{0.68,0.00,0.00}{#1}}
\newcommand{\FunctionTok}[1]{\textcolor[rgb]{0.28,0.35,0.67}{#1}}
\newcommand{\ImportTok}[1]{\textcolor[rgb]{0.00,0.46,0.62}{#1}}
\newcommand{\InformationTok}[1]{\textcolor[rgb]{0.37,0.37,0.37}{#1}}
\newcommand{\KeywordTok}[1]{\textcolor[rgb]{0.00,0.23,0.31}{#1}}
\newcommand{\NormalTok}[1]{\textcolor[rgb]{0.00,0.23,0.31}{#1}}
\newcommand{\OperatorTok}[1]{\textcolor[rgb]{0.37,0.37,0.37}{#1}}
\newcommand{\OtherTok}[1]{\textcolor[rgb]{0.00,0.23,0.31}{#1}}
\newcommand{\PreprocessorTok}[1]{\textcolor[rgb]{0.68,0.00,0.00}{#1}}
\newcommand{\RegionMarkerTok}[1]{\textcolor[rgb]{0.00,0.23,0.31}{#1}}
\newcommand{\SpecialCharTok}[1]{\textcolor[rgb]{0.37,0.37,0.37}{#1}}
\newcommand{\SpecialStringTok}[1]{\textcolor[rgb]{0.13,0.47,0.30}{#1}}
\newcommand{\StringTok}[1]{\textcolor[rgb]{0.13,0.47,0.30}{#1}}
\newcommand{\VariableTok}[1]{\textcolor[rgb]{0.07,0.07,0.07}{#1}}
\newcommand{\VerbatimStringTok}[1]{\textcolor[rgb]{0.13,0.47,0.30}{#1}}
\newcommand{\WarningTok}[1]{\textcolor[rgb]{0.37,0.37,0.37}{\textit{#1}}}

\providecommand{\tightlist}{%
  \setlength{\itemsep}{0pt}\setlength{\parskip}{0pt}}\usepackage{longtable,booktabs,array}
\usepackage{calc} % for calculating minipage widths
% Correct order of tables after \paragraph or \subparagraph
\usepackage{etoolbox}
\makeatletter
\patchcmd\longtable{\par}{\if@noskipsec\mbox{}\fi\par}{}{}
\makeatother
% Allow footnotes in longtable head/foot
\IfFileExists{footnotehyper.sty}{\usepackage{footnotehyper}}{\usepackage{footnote}}
\makesavenoteenv{longtable}
\usepackage{graphicx}
\makeatletter
\def\maxwidth{\ifdim\Gin@nat@width>\linewidth\linewidth\else\Gin@nat@width\fi}
\def\maxheight{\ifdim\Gin@nat@height>\textheight\textheight\else\Gin@nat@height\fi}
\makeatother
% Scale images if necessary, so that they will not overflow the page
% margins by default, and it is still possible to overwrite the defaults
% using explicit options in \includegraphics[width, height, ...]{}
\setkeys{Gin}{width=\maxwidth,height=\maxheight,keepaspectratio}
% Set default figure placement to htbp
\makeatletter
\def\fps@figure{htbp}
\makeatother

\usepackage{booktabs}
\usepackage{longtable}
\usepackage{array}
\usepackage{multirow}
\usepackage{wrapfig}
\usepackage{float}
\usepackage{colortbl}
\usepackage{pdflscape}
\usepackage{tabu}
\usepackage{threeparttable}
\usepackage{threeparttablex}
\usepackage[normalem]{ulem}
\usepackage{makecell}
\usepackage{xcolor}
\usepackage{fvextra}
\DefineVerbatimEnvironment{Highlighting}{Verbatim}{breaklines,commandchars=\\\{\}}
\DefineVerbatimEnvironment{OutputCode}{Verbatim}{breaklines,commandchars=\\\{\}}
\KOMAoption{captions}{tableheading}
\makeatletter
\makeatother
\makeatletter
\makeatother
\makeatletter
\@ifpackageloaded{caption}{}{\usepackage{caption}}
\AtBeginDocument{%
\ifdefined\contentsname
  \renewcommand*\contentsname{Table of contents}
\else
  \newcommand\contentsname{Table of contents}
\fi
\ifdefined\listfigurename
  \renewcommand*\listfigurename{List of Figures}
\else
  \newcommand\listfigurename{List of Figures}
\fi
\ifdefined\listtablename
  \renewcommand*\listtablename{List of Tables}
\else
  \newcommand\listtablename{List of Tables}
\fi
\ifdefined\figurename
  \renewcommand*\figurename{Figure}
\else
  \newcommand\figurename{Figure}
\fi
\ifdefined\tablename
  \renewcommand*\tablename{Table}
\else
  \newcommand\tablename{Table}
\fi
}
\@ifpackageloaded{float}{}{\usepackage{float}}
\floatstyle{ruled}
\@ifundefined{c@chapter}{\newfloat{codelisting}{h}{lop}}{\newfloat{codelisting}{h}{lop}[chapter]}
\floatname{codelisting}{Listing}
\newcommand*\listoflistings{\listof{codelisting}{List of Listings}}
\makeatother
\makeatletter
\@ifpackageloaded{caption}{}{\usepackage{caption}}
\@ifpackageloaded{subcaption}{}{\usepackage{subcaption}}
\makeatother
\makeatletter
\@ifpackageloaded{tcolorbox}{}{\usepackage[skins,breakable]{tcolorbox}}
\makeatother
\makeatletter
\@ifundefined{shadecolor}{\definecolor{shadecolor}{rgb}{.97, .97, .97}}
\makeatother
\makeatletter
\makeatother
\makeatletter
\makeatother
\ifLuaTeX
  \usepackage{selnolig}  % disable illegal ligatures
\fi
\IfFileExists{bookmark.sty}{\usepackage{bookmark}}{\usepackage{hyperref}}
\IfFileExists{xurl.sty}{\usepackage{xurl}}{} % add URL line breaks if available
\urlstyle{same} % disable monospaced font for URLs
\hypersetup{
  pdftitle={{[}Your Informative Title Here{]}},
  pdfauthor={Sirohi Kumar and Chaira Harder},
  colorlinks=true,
  linkcolor={blue},
  filecolor={Maroon},
  citecolor={Blue},
  urlcolor={Blue},
  pdfcreator={LaTeX via pandoc}}

\title{{[}Your Informative Title Here{]}}
\author{Sirohi Kumar and Chaira Harder}
\date{Invalid Date}

\begin{document}
\maketitle
\ifdefined\Shaded\renewenvironment{Shaded}{\begin{tcolorbox}[borderline west={3pt}{0pt}{shadecolor}, boxrule=0pt, interior hidden, breakable, sharp corners, frame hidden, enhanced]}{\end{tcolorbox}}\fi

\emph{Your written short report should be the first thing your reader
encounters in your case study document. While you may choose to label
each of the required components/sections of the short report with its
own header (see below for an example of creating a header in Quarto),
you do not need to do so provided that all of the required information
is included. Your report should be self-contained and written in such a
way that a quantitatively inclined friend (who has taken or is taking
SDS 291) could follow what you did without necessarily knowing anything
about the 2000 presidential election otherwise.}

\emph{As you write your report, you may wish to reference the guide to
typesetting regression lines in Quarto using LaTeX (linked at the top of
our class Moodle page), the Quarto help page for formatting documents
using Markdown (), and the Quarto help page for customizing the output
from executed code chunks.}

\begin{center}\rule{0.5\linewidth}{0.5pt}\end{center}

\hypertarget{introduction}{%
\section{INTRODUCTION}\label{introduction}}

\hypertarget{results}{%
\section{RESULTS}\label{results}}

\hypertarget{generate-data}{%
\subsection{Generate Data}\label{generate-data}}

\hypertarget{load-libraries}{%
\subsubsection{load libraries}\label{load-libraries}}

\begin{Shaded}
\begin{Highlighting}[]
\FunctionTok{library}\NormalTok{(tidyverse)}
\FunctionTok{library}\NormalTok{(Sleuth2) }
\FunctionTok{library}\NormalTok{(kableExtra)  }
\FunctionTok{library}\NormalTok{(broom)}
\FunctionTok{library}\NormalTok{(performance)}
\end{Highlighting}
\end{Shaded}

\hypertarget{data-wrangling}{%
\subsubsection{data wrangling}\label{data-wrangling}}

\emph{include paragraph on data wrangling here}

\begin{Shaded}
\begin{Highlighting}[]
\CommentTok{\# Loading the case study data}
\NormalTok{election }\OtherTok{\textless{}{-}}\NormalTok{ Sleuth2}\SpecialCharTok{::}\NormalTok{ex0825}

\CommentTok{\# Creating a second dataset with Palm Beach County excluded}
\NormalTok{election\_wo\_pb }\OtherTok{\textless{}{-}}\NormalTok{ election }\SpecialCharTok{|\textgreater{}} \FunctionTok{filter}\NormalTok{(County }\SpecialCharTok{!=} \StringTok{"Palm Beach"}\NormalTok{)}
\end{Highlighting}
\end{Shaded}

\hypertarget{explore-data}{%
\subsection{Explore Data}\label{explore-data}}

\emph{describe the data layout + clearly outline the explanatory and
response variables}

\includegraphics{sds-291_s-24_case-study-template_files/figure-pdf/unnamed-chunk-3-1.pdf}

\hypertarget{find-most-appropriate-model}{%
\subsection{Find Most Appropriate
Model}\label{find-most-appropriate-model}}

\hypertarget{generate-various-models}{%
\subsubsection{Generate various models}\label{generate-various-models}}

\emph{expand on what we're doing and why we're doing it}

\hypertarget{r-squared-values}{%
\subsubsection{R-Squared values}\label{r-squared-values}}

We can now compare the \(R^2\) values of these various transformations,
and from this table we know that the \texttt{ln(Buchanan)\ vs\ ln(Bush)}
model has the highest correlation.

\begin{longtable}[]{@{}
  >{\raggedleft\arraybackslash}p{(\columnwidth - 6\tabcolsep) * \real{0.2000}}
  >{\raggedleft\arraybackslash}p{(\columnwidth - 6\tabcolsep) * \real{0.2500}}
  >{\raggedleft\arraybackslash}p{(\columnwidth - 6\tabcolsep) * \real{0.2500}}
  >{\raggedleft\arraybackslash}p{(\columnwidth - 6\tabcolsep) * \real{0.3000}}@{}}
\caption{R-Squared (correlation) values for various
models}\tabularnewline
\toprule\noalign{}
\begin{minipage}[b]{\linewidth}\raggedleft
Buchanan v Bush
\end{minipage} & \begin{minipage}[b]{\linewidth}\raggedleft
ln(Buchanan) v Bush
\end{minipage} & \begin{minipage}[b]{\linewidth}\raggedleft
Buchanan v ln(Bush)
\end{minipage} & \begin{minipage}[b]{\linewidth}\raggedleft
ln(Buchanan) v ln(Bush)
\end{minipage} \\
\midrule\noalign{}
\endfirsthead
\toprule\noalign{}
\begin{minipage}[b]{\linewidth}\raggedleft
Buchanan v Bush
\end{minipage} & \begin{minipage}[b]{\linewidth}\raggedleft
ln(Buchanan) v Bush
\end{minipage} & \begin{minipage}[b]{\linewidth}\raggedleft
Buchanan v ln(Bush)
\end{minipage} & \begin{minipage}[b]{\linewidth}\raggedleft
ln(Buchanan) v ln(Bush)
\end{minipage} \\
\midrule\noalign{}
\endhead
\bottomrule\noalign{}
\endlastfoot
0.7517819 & 0.6790001 & 0.571179 & 0.8658343 \\
\end{longtable}

\hypertarget{visualize-new-model}{%
\subsubsection{Visualize new model}\label{visualize-new-model}}

\emph{describe the spread of the data in this version}

\includegraphics{sds-291_s-24_case-study-template_files/figure-pdf/unnamed-chunk-6-1.pdf}

\hypertarget{goodness-of-fit}{%
\subsection{Goodness of Fit}\label{goodness-of-fit}}

\hypertarget{residuals}{%
\subsubsection{Residuals}\label{residuals}}

Our selected model, in which both the Bush and Buchanan votes are
logged, yields the greatest \(R^2\) value at \(0.8658343\). With its
\(R^2\) value, it captures a stronger linear relationship than its
comparison models used (Buchanan vs Bush, \(ln(Buchanan)\) vs Bush,
Buchanan vs \(ln(Bush)\)) and also suggests a more proportional scaling
between the votes, as we can see in the visual above.

To see if our our model is truly effective and the votes in the dataset
are not random, we can use the Linearity test and examine the Normality
of Residuals.

\hypertarget{linearity-test}{%
\paragraph{Linearity Test}\label{linearity-test}}

\includegraphics{sds-291_s-24_case-study-template_files/figure-pdf/unnamed-chunk-7-1.pdf}

\includegraphics{sds-291_s-24_case-study-template_files/figure-pdf/unnamed-chunk-7-2.pdf}

In this case, our reference line, in green, is not perfectly flat and
horizontal, thus it does not pass the linearity test. But it's not bad
tho\ldots{}

\hypertarget{homogeneity-of-variance}{%
\paragraph{Homogeneity of variance}\label{homogeneity-of-variance}}

\includegraphics{sds-291_s-24_case-study-template_files/figure-pdf/unnamed-chunk-8-1.pdf}

\hypertarget{normality-of-residuals}{%
\paragraph{Normality of Residuals}\label{normality-of-residuals}}

\includegraphics{sds-291_s-24_case-study-template_files/figure-pdf/unnamed-chunk-9-1.pdf}

\includegraphics{sds-291_s-24_case-study-template_files/figure-pdf/unnamed-chunk-9-2.pdf}

Sources:
http://www.sthda.com/english/articles/39-regression-model-diagnostics/161-linear-regression-assumptions-and-diagnostics-in-r-essentials/

\hypertarget{predictions}{%
\subsection{Predictions}\label{predictions}}

\hypertarget{regression-line}{%
\subsubsection{Regression Line}\label{regression-line}}

Let \(Bush_i\) denote the \(ln\) number of Bush votes in any Florida
county during the 2000 election. Using the regression model from above,
we can predict \(Buchanan_i\), the \(ln\) of the number of Buchanan
votes in any Florida county \(i\) (during the 2000 election) based on
any \(Bush_i\) value. Using this regression model, we can find
\[Buchanan_i = \beta_0 + \beta_1\left(Bush_i\right).\]

\begin{table}[H]
\centering
\begin{tabular}[t]{lcccc}
\toprule
  & Estimate & Std. Error & t value & p value\\
\midrule
(Intercept) & -2.3415 & 0.3544 & -6.6066 & 0\\
log(Bush2000) & 0.7310 & 0.0360 & 20.3229 & 0\\
\bottomrule
\end{tabular}
\end{table}

This linear model predicts that for each increase in \(ln(Bush_i)\),
that \(ln(Buchanan_i)\) should increase by \(0.7310\) votes. We can now
use this model to predict the number of votes Buchanan should have
received in Palm Beach County, if Palm Beach county was the same as the
other Florida counties.

\hypertarget{prediction-interval}{%
\subsubsection{Prediction Interval}\label{prediction-interval}}

\textbf{Obtain a 95\% prediction interval for the number of Buchanan
votes in Palm Beach from this result---assuming the relationship is the
same in this county as in the others}

\begin{table}[H]
\centering
\begin{tabular}[t]{ccc}
\toprule
center & lower & upper\\
\midrule
6.384143 & 5.524656 & 7.24363\\
\bottomrule
\end{tabular}
\end{table}

We can transform these numbers to determine the non-\(ln\) vote count.

lower: \(e^{5.524656} = 250.8\) upper: \(e^{7.24363} = 1399.164\)

Our prediction interval tells us that, based on this sample, 95\% of the
time, the number of votes for Buchanan in the 2000 election should be
between 250.8 and 1399.164 votes, given that 152,846 people voted for
Bush.

However, in the 2000 election, Buchanan received 3407 votes, which is
over twice as large as the upper bound of our interval. This shows that
Palm Beach county's votes for Bush and Buchanan are highly irregular
compared to other Florida counties.

\hypertarget{gores-votes}{%
\subsection{Gore's Votes}\label{gores-votes}}

\textbf{Assuming that some of the votes cast for Buchanan were intended
as votes for Gore, use the prediction interval to give an estimate for
the likely number of votes intended for Gore but cast for Buchanan.}

There were 3407 votes for Buchanan in the 2000 election, but our
prediction interval tells us that the number of votes expected for
Buchanan 95\% of the time is between 250.800 and 1399.164, so the likely
number of votes intended for Gore but cast for Buchanan should be
between \(3407-250.800 = 3156.200\) and \(3407-1399.164 = 2007.836\).

\hypertarget{discussion}{%
\section{DISCUSSION}\label{discussion}}

Our goal with this case study was to determine if, based on the vote
counts for Pat Buchanan and George Bush in every Florida county, we
could conclude whether Buchanan received an unusual number of votes in
Palm Beach county. We used a linear model on data that had been
transformed to show that for every increase in \(ln(Bush_i)\) by one
\(ln(Buchanan_i)\) should increase by about \(0.7310\) votes.

We used this model to generate a prediction interval that predicted with
95\% confidence that Buchanan's vote count in Palm Beach County should
have been between \(250.8\) and \(1399.164\). Instead, Buchanan received
\(3407\) votes, a highly irregular value, according to this model. Based
on this, we can conclude that there is evidence Buchanan received an
unusually high number of votes in Palm Beach county.

This deviation from the number of expected votes lends credence to the
complaints of many Democratic voters who reported having accidentally
voted for Buchanan (the Reform candidate) instead of Al Gore (the
Democratic candidate), because of the confusing ballot layout. On a
larger scale, this shows that the incredibly close election of George
Bush in 2000 may have been -- at least in part -- due to a fluke in
ballot design.

However, we can't make a conclusive claim that this is the case. First
of all, our model only calculates the correlation between Buchanan and
Bush's votes -- it cannot determine if there's a causal relationship
between the two, or indeed the existence of any causal factors affecting
the relationship between the two. Additionally, we cannot directly
attribute the unusual number of votes for Buchanan to the ballot layout,
as we haven't examined any other elections with strange ballots and,
again, this is not a causal model.

\begin{center}\rule{0.5\linewidth}{0.5pt}\end{center}

\emph{When you create plots for your case study report, the
\texttt{echo:\ false} chunk option tells Quarto to include the final
output of your R commands (in this case, a plot) in your rendered PDF
without} printing the underlying R commands that generated that plot!
The message and warning flags both prevent R from printing any
additional text with error messages or warnings to the PDF.

\begin{Shaded}
\begin{Highlighting}[]
\CommentTok{\# \#| echo: false}
\CommentTok{\# }
\CommentTok{\# \# Fitting the regression line for mean mortality as a function of wine consumption}
\CommentTok{\# lm.wine \textless{}{-} lm(Mortality \textasciitilde{} Wine, data = wine)}
\CommentTok{\# }
\CommentTok{\# \# Representing the regression table as a dataframe (i.e., tidying the summary() output)}
\CommentTok{\# lm.wine.table \textless{}{-} summary(lm.wine)$coefficients}
\CommentTok{\# }
\CommentTok{\# \# Creating a nicely formatted table from the dataframe using the kable package}
\CommentTok{\# \#   You can find more information about this package here: https://haozhu233.github.io/kableExtra/awesome\_table\_in\_pdf.pdf}
\CommentTok{\# lm.wine.table |\textgreater{} kbl(col.names = c("Name for Col. 1", "Name for Col. 2", "Name for Col. 3", "Name for Col. 4"), align = "c", booktabs = T, linesep="", digits = c(2, 2, 2, 4)) |\textgreater{} kable\_classic(full\_width = F, latex\_options = c("HOLD\_position"))}
\end{Highlighting}
\end{Shaded}

\hypertarget{r-appendix}{%
\section{R APPENDIX}\label{r-appendix}}

\emph{Copy and paste all code that you used for your case study into one
chunk at the end of your written report. Before submitting your case
study, take one final look at the R Appendix and make sure that all code
is clearly visible. If you see a line running off the side of the PDF,
please split the code over multiple lines using a linebreak.}

\begin{Shaded}
\begin{Highlighting}[]
\CommentTok{\# \#| message: FALSE}
\CommentTok{\# \#| warning: FALSE}
\CommentTok{\# }
\CommentTok{\# \# Loading necessary packages}
\CommentTok{\# library(tidyverse)}
\CommentTok{\# library(Sleuth2)}
\CommentTok{\# library(broom)        }
\CommentTok{\# library(kableExtra)   }
\CommentTok{\# }
\CommentTok{\# \# Loading the case study data}
\CommentTok{\# election \textless{}{-} Sleuth2::ex0825}
\CommentTok{\# }
\CommentTok{\# \# Creating a second dataset with Palm Beach County excluded}
\CommentTok{\# election\_wo\_pb \textless{}{-} election |\textgreater{} filter(County != "Palm Beach")}
\CommentTok{\# }
\CommentTok{\# \# Loading another dataset on wine consumption and heart disease mortality}
\CommentTok{\# wine \textless{}{-} Sleuth2::ex0823}
\CommentTok{\# }
\CommentTok{\# \# Creating a scatterplot for the relationship between mortality and wine consumption}
\CommentTok{\# wine |\textgreater{} ggplot(aes(x = Wine, y = Mortality)) + geom\_point() + }
\CommentTok{\#   ggtitle("Association between wine consumption and mortality rates.")}
\CommentTok{\# }
\CommentTok{\# \# Fitting and summarizing the regression line for mean mortality }
\CommentTok{\# \# as a function of wine consumption}
\CommentTok{\# wine.lm \textless{}{-} lm(Mortality \textasciitilde{} Wine, data = wine)}
\CommentTok{\# wine.lm.table \textless{}{-} wine.lm |\textgreater{} tidy()}
\CommentTok{\# wine.lm.table |\textgreater{} kbl(col.names = c("Name for Col. 1", "Name for Col. 2", }
\CommentTok{\#                                    "Name for Col. 3", "Name for Col. 4", }
\CommentTok{\#                                    "Name for Col. 5"), }
\CommentTok{\#                      align = "c", booktabs = T, linesep="", }
\CommentTok{\#                      digits = c(2, 2, 2, 4)) |\textgreater{} }
\CommentTok{\#   kable\_classic(full\_width = F, latex\_options = c("HOLD\_position"))}
\end{Highlighting}
\end{Shaded}




\end{document}
